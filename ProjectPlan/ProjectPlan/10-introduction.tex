\chapter{Introduction}
This document is the project plan for the creation of a demo facility for the LOGwear project.

\section{LOGwear}
LOGwear is a research project that aims to bring wearable to the area of logistics, especially to \gls{sme}. It is a German-Dutch project were multiple parties are cooperating to create results in the research project. Involved in this are two Universities of applied sciences, \gls{fhtenl} in Venlo as the Lead-Partner, Netherlands and \gls{hsnr} in Krefeld, Germany.

Further on there are also multiple partner companies involved in the project namely KLG Europe bv, Helmut Beyers GmbH and imat-uve GmbH. These partner companies are there to give the knowledge about logistics processes, as well as to verify and test the results.

The project is backed within the scope of the INTERREG Deutschland-Nederland initiative. It is backed by the \gls{eu}, \gls{mweimh} and the Provincie Limburg as well.

\section{Demo Facility}
The demo facility that should be created for this project will be a physical environment in which a logistics process can be modelled. This process should then be improved by using a wearable. This will be used as a demonstration area for \gls{sme} to see hands-on if a wearable could be used to improve their own processes. The demo facility will be a proof of concept and not a fully implemented solution that could be directly used at a logistics company and instantly work. %This document will inform the user about the LOGwear project in general and more specifically about the task of creating a demo facility for that project and the boundaries of that task.  The demo facility in this is supposed to show the capabilities of a wearable by displaying it is an actual scenario. What this means is a general logistics process is taken and improved with a wearable and then displayed to interested \gls{sme}. It is supposed to be a proof of concept and not a fully implemented solution that could be directly used at a logistics company and instantly work.
\section{Overview}
In this section the following chapters will be shortly explained. 

The first chapter following this, chapter \ref{cha:scope} is about the scope of the assignment as well as the way of working. The risk management done in the project will be explained in chapter \ref{cha:riskManagement}. The last chapter will be about the existing stakeholders in the project \ref{cha:stakeholders}.