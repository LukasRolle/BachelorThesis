\chapter{Introduction}\pagenumbering{arabic}
This document is the project plan for the creation of a demo facility for the LOGwear project.

\section{LOGwear}

\noindent LOGwear is a research project that aims to bring wearable to the area of logistics, especially to \gls{sme}. It is a German-Dutch project were multiple parties are cooperating to create results in the research project. Involved in this are two Universities of applied sciences, \gls{fhtenl} in Venlo as the Lead-Partner, Netherlands and \gls{hsnr} in Krefeld, Germany.

Further on there are also multiple partner companies involved in the project namely KLG Europe bv, Helmut Beyers GmbH and imat-uve GmbH. These partner companies are there to give the knowledge about logistics processes, as well as to verify and test the results.

The project is backed within the scope of the INTERREG Deutschland-Nederland initiative. It is backed by the \gls{eu}, \gls{mweimh} and the Provincie Limburg as well.



\section{Demo Facility}
The demo facility that should be created for this project will be a physical environment in which a logistics process can be modelled. This process should then be improved by using a wearable. This will be used as a demonstration area for \gls{sme} to see hands-on if a wearable could be used to improve their own processes. The demo facility will be a proof of concept and not a fully implemented solution that could be directly used at a logistics company and instantly work.

\section{Schedule}
The initial schedule for the project can be seen in table \ref{tab:schedule}. It is to be mentioned, that the schedule is subject to change as the project goes on. The project will be executed in a scrum-like way that is adapted to the group, given the group size of two developers.
\begin{table}[htbp]
\centering
\large
\resizebox{1\textwidth}{!} {
\begin{tabular}{|c"cc:cccc:cc:ccccccccc:ccc:c|} \hline
\textbf{Sprints} & \multicolumn{2}{>{\centering\arraybackslash} m{.15\textwidth}|}{\textbf{Logistics Processes \& Wearables}} & \multicolumn{4}{>{\centering\arraybackslash} m{.35\textwidth}|}{\textbf{Reference Architecture}} & \multicolumn{2}{>{\centering\arraybackslash} m{.25\textwidth}|}{\textbf{Research Demo Facility}} & \multicolumn{9}{c|}{\textbf{Demo Facility Design and Implementation}}                   & \multicolumn{3}{>{\centering\arraybackslash} m{.25\textwidth}|}{\textbf{Creation Demo Facility}} & \multicolumn{1}{>{\centering\arraybackslash} m{.08\textwidth}|}{\textbf{Buf-fer}} \\ \thickhline
Date                 & 06.02              & 13.02              & 20.02                & 27.02                & 06.03         & 13.03        & 20.03        & 27.03 & 03.04 & 10.04 & 17.04 & 24.04 & 01.05 & 08.05 & 15.05 & 22.05 & 29.05 & 05.06         & 12.06        & 19.06        & 26.06                             \\
Week                       & 6                  & 7                  & 8                    & 9                    & 10            & 11           & 12           & 13    & 14    & 15    & 16    & 17    & 18    & 19    & 20    & 21    & 22    & 23            & 24           & 25           & 26 \\\hline       	              
\end{tabular}
}
\caption{Schedule}
\label{tab:schedule}
\end{table}

The schedule is divided in work packages that are to be executed, it is to be noted, that each work package could be split into multiple sprints in the future. In the following subsections the work packages will be explained.

\subsection{Logistics Processes \& Wearables}
This work package includes research about the gives processes and wearables in general, as well as already choosing potential wearables that could be used to improve the process. The end result for this should be a decision on process and wearable. But the result for this could potentially take longer than this task is scheduled. The wearables should be ranked after getting hands-on experience on them, therefore some of them have to be ordered first.

\subsection{Reference Architecture}
A reference architecture should be created for a sample wearable application. What this work package contains is, the creation of diagrams that show the communication from a wearable to the \gls{wms} or something similar. What should not be created is a full reference architecture for a process that is implemented with a concrete wearable. 

It is about creating the always needed layers when using a wearable in a way that supports most wearable solutions.

\subsection{Research Demo Facility}
This task includes researching what physical objects and what systems would be needed to create a demo facility that could showcase a single process with a single wearable. This also includes the gathering of knowledge of where the demo facility should be created and where to get the needed objects.

\subsection{Demo Facility Design and Implementation}
The work package includes the creation of the software design and implementation for the wearable and all aspects that are needed to fully showcase a process.

\subsection{Creation Demo Facility}
This task includes the physical creation of the demo facility. This means setting up shelves with packages to scan and put on a hand pallet truck. Setting up barcodes on the packages to scan. Setting up an environment that can showcase what is happening better to an audience.


\section{Overview}
In this section the following chapters will be shortly explained. 

The first chapter following this, chapter \ref{cha:scope} is about the scope of the assignment. The risk management done in the project will be explained in chapter \ref{cha:riskManagement}. The last chapter will be about the existing stakeholders in the project \ref{cha:stakeholders}.