\chapter{Scope}\label{cha:scope}
The scope management is about setting boundaries for the project, which means clearly defining what should be done in the project and, more importantly, what is not part of the project. In section \ref{sec:insideScope} it will be discussed what should be created during the project, while section \ref{sec:outsideScope} is about what should not be created.

\section{Inside of the Scope}\label{sec:insideScope}
\begin{description}
	\item[Reference Architecture] \hfill
	
	A reference architecture for a wearable will be designed and implemented. The reference architecture should allow for simpler future implementations, of applications that should support logistics processes. This reference architecture gives the basic packages that need to exist when working in a wearable in a logistics environment. The reference architecture is given with design, a basic implementation and documentation on different levels of abstractness \cite{Kruchten:1995}.
	\item[Prototype Wearable Application] \hfill
	
	A prototype for a wearable application. It should be a implemented and working solution that is supporting a logistics process. The application is supposed to be a proof on concept.
	\item[Creation Physical Demo Area] \hfill
	
	An area should be rented and prepared to be able to show the prototype application to interested companies. This is intended as a possibility for \gls{sme} to hands-on experience what a wearable could do to improve their processes.
\end{description}

\section{Outside of the Scope}\label{sec:outsideScope}
\begin{description}
	\item[Finished Product] \hfill
	
	At the end of the project, there should not be a finished product that can be dropped in at any logistics company. Connections to a \gls{wms} or similar should be mocked, but designed to be replaceable to give \gls{sme} a widely usable reference architecture.
	\item[Multiple Demo Scenarios] \hfill
	
	The aim of this project is not to implement multiple demo scenarios, but one more sophisticated process that should be modelled. 
\end{description}