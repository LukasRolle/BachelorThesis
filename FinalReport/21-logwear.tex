\section{LOGwear}\label{sec:logwear}
LOGwear is a research project that aims to bring wearables to the area of logistics, especially to \gls{sme}. It is a German-Dutch research project were multiple parties are cooperating to create results. Involved in this are two Universities of applied sciences, \gls{fhtenl} in Venlo as the lead partner, Netherlands and \gls{hsnr} in Krefeld, Germany.

Further on there are also multiple partner companies involved in the project, namely KLG Europe bv, Helmut Beyers GmbH and imat-uve GmbH. These partner companies are there to give the knowledge about logistics processes, as well as to verify and test the results.

The project is backed within the scope of the INTERREG Deutschland-Nederland initiative. It is backed by the \gls{eu}, \gls{mweimh} and the Provincie Limburg as well.

The LOGwear project is consisting of three main \gls{wp}s. \citep{website:logwear}
\begin{description}
	\item[Knowledge Base] \hfill \\
		The knowledge base is a platform that allows to exchange information between logistics companies which wearable can be used for which process. \citep{bachelorThesis:oliver} \citep{bachelorThesis:sascha}
	\item[Reference Architecture / Reference Model] \hfill \\
		Even though the official site still states, that the expected result for \gls{wp} 2 is a reference architecture, the expected result has internally already changed to a reference model, further detail about what is expected from the reference model can be found in chapter \ref{cha:reference}.
	\item[Demo Facility] \hfill \\
		The demo facility is the creation of a physical demo that allows \gls{sme} to see the benefits of wearables for a demo process. Further details for the demo facility are described in chapter \ref{cha:demoFacility}.
\end{description}