\chapter{Introduction}
This thesis is written as the completion of the study course software engineering at the Fontys Hogeschool Techniek en Logistiek in Venlo, Netherlands. The graduation project is conducted at the LOGwear research project, the research project itself will be explained in section \ref{sec:logwear}. The thesis is written over the course of five months and is documenting the thought and creation process of the project, of creating a demo facility for the usage of a wearable in a logistics process.

This report is written during the second month of the thesis, therefore a lot of the tasks that should be done are not finished and it can only be explained how they are planned.

As this thesis was written on a software engineering topic, pieces of code will occur throughout the thesis, single words, like variables, or classes will be written in a \texttt{mono-spaced-font}. 

Furthermore the word \gls{package} in this report is always meant as a software package and never as the package that is used in the logistics sector, the word \gls{parcel} is used in this context.
\section*{Overview}
The following chapters in this report will contain these topics:

\begin{description}
	\item[Context and Scope] \hfill \\
	In chapter \ref{cha:context} the context of the project will be elaborated, naming the involved parties and what the research project is about. Furthermore the scope of the thesis will be defined, including demarcation. The general information, including project management details are explained.
	\item[Research] \hfill \\
	In chapter \ref{cha:research} the general research part that was done during the thesis will be explained. The results of the research will be named and the chosen \gls{case} tools listed.
	\item[Reference Architecture] \hfill \\
	Chapter \ref{cha:reference} contains the design process and the connected problems with the reference model.
	\item[Demo Facility] \hfill \\
	Chapter \ref{cha:demoFacility} will contain the infrastructure, design and implementation of the demo facility that is to be created to showcase the possibilities of wearables in the area of logistics.
	\item[Conclusion] \hfill \\
	Chapter \ref{cha:conclusion} will contain the conclusion, reflection and a look into the future, showing how the project is planned to develop.
\end{description}