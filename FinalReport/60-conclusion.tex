\chapter{Conclusion}\label{cha:conclusion}
Up to this point, the reference model has been fully designed and is documented with multiple variations for different situations. The database for the demo has been deployed on an azure cloud service. The \gls{rest} interface to that database is deployed on the same azure instance. The web application is still being developed at this point, as well as the connection from the wearable to this web application. The facility area has also not been rented and filled at this point. 

Concluding the whole tasks that are to be finished at the end of this project is, that the research process for the available wearables is finished, but more wearables are going to be available in the future for the logwear project. Those wearables will be tested and compared in the future, but that will most likely not be a part of this project anymore. The requirements analysis and design phase for the reference model and the demo facility are also completely finished at this point. The infrastructure surrounding the demo facility is fully implemented as well. The two parts that are not fully implemented at this point are the web application and the wearable application itself. While there are still tasks to complete when that is finished as well, these are the main tasks that are a part of this project.



%The reference model is the only current deliverable really created up to this point. This is due to the reference model being a crucial point of the research project. As the research goes on the reference model will be used as the artifact that is used to show developers on how a wearable system should look like in a logistics environment. Therefore the design went through multiple iterations to allow it to fit as many use-cases as possible, while reducing the need to change a lot of the design for each use-case.

%The design of the reference model, even though a few communicational problems arose with what was actually expected from it, was successful and the involved parties are happy with how it turned out.

\section{Recommendations}\label{sec:recommendations}
The recommendations are towards possibilities of the project in the future when the initial project of creating a demo facility is done.

There is still work to be done in the area of the demo facility after the initial one is done. There might be an internship or a bachelor thesis that is concerned with creating demos for each of the wearables available to the logwear project. The focus here could be for the research part on the possibilities of the different wearables for the same task. The information gained from the different demos can be used to compare the different wearables more effectively and on a level that was not possible beforehand. The design part could go more into detail on how the different wearables and environments are requiring different design patterns and ideas. The implementation is rather interesting again due to the multitude of environments and technologies used in each wearable. This type of project could be especially interesting for \gls{sme} so that they get the ability to test a lot of different wearables out themselves.

Another interesting project for an internship or bachelor student could be the implementation of a process improved by wearables at a pilot company. The emphasis there could be on the differences between a sandbox style demo and a demo that has all the constraints that are imposed by the environment of the pilot company.


\section{Further Planning}
Apart from the recommendations given in section \ref{sec:recommendations}, this project still has a few tasks left, that will be handled. Further functionalities that might still be added to the web application:

\begin{itemize}
	\item Dividing a room in multiple sectors to help new picking workers to more smoothly get used to their work. And help a worker to find the most optimal route through the warehouse.
	\item Implementing a mechanism that is helping the picking worker to confirm the quantity.
\end{itemize}

Apart from different functionalities that can still be added to the demo facility, there is still other tasks to be done. Further documentation for a handover and a document on how to set up the existing system on other systems. The area for the demo environment needs to be prepared before the demo can actually be showcased.

When all that is done, the demo can still be actually showcased to interested \gls{sme}, to show the advantages of wearables in the order picking process.
