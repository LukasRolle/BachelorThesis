\chapter{Conclusion}\label{cha:conclusion}
The reference model is the only current deliverable really created up to this point. This is due to the reference model being a crucial point of the research project. As the research goes on the reference model will be used as the artifact that is used to show developers on how a wearable system should look like in a logistics environment. Therefore the design went through multiple iterations to allow it to fit as many use-cases as possible, while reducing the need to change a lot of the design for each use-case.

The design of the reference model, even though a few communicational problems arose with what was actually expected from it, was successful and the involved parties are happy with how it turned out.

\section*{Further Planning}
Now that the design of the reference model is at a stable state, it can be started to implement the parts of it that are not directly relying on the wearable itself. The wearables will be tested as they become available and a decision will be made depending on the results from that testing. Once that is done, the further creation of the demo facility will be able to planned accordingly and implementation of that can start.

When the creation of the demo facility is finished it is planned to invite \gls{sme}s to test out the wearables and how they could improve their processes.