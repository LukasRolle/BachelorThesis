\documentclass[a4paper]{report}

\usepackage[parfill]{parskip}
\usepackage[pdftex, scale={.8,.8}]{geometry}
\usepackage{titlesec}
\titleformat{\chapter}[hang]
{\normalfont\huge\bfseries}{\thechapter}{1em}{}
\titlespacing*{\chapter}{0pt}{0pt}{20pt}
\begin{document}

\chapter*{Reflection \normalsize{Lukas Rolle - 2310309}}
This reflection is written at the time where the final report of my bachelor thesis is to be handed, and I will reflect on my personal, professional growth and problem solving process in regards of the way of working. Furthermore on the personal problems that occurred during the report and how they were handled.

The bachelor thesis was organized in an open way, meaning that the tasks that were to be done had to be acquired by myself and a priority was to be added to them by myself as well. This was true for the beginning of the thesis up to about the halfway point, where the mid-term report was to be delivered. This was not the most comfortable working environment for me, as I was used to getting a more concrete task to work on. During the time I got more used to that and learned to enjoy the freedom that was connected with that, but also realized the boundaries and burdens of such an approach. I think it is a nice approach to get creative and innovative ideas and solutions, but it is not necessarily the approach that is creating the most productive environment. At least I am being most productive in an environment where a clear goal is set from the beginning, and milestones are known from the beginning. The beginning of the task was overwhelming in the variety of tasks that could be worked on and as time progressed more tasks revealed themselves. This state continued until about the mid-term report. 

For the mid-term report the boundaries were set and a lot of the smaller tasks that showed themselves were finished. This allowed for a more focussed way of working and also an environment that I was more comfortable with.

So concluding the first half of the bachelor thesis, it was an environment, that I was not completely comfortable with, but the experiences made with such an environment was interesting and I will take the lesson learned about my own way of working and will try to improve my way of handling such a situation when it occurs again in the future.

Following the mid term presentation I caught the flu and was ill for two weeks and am still struck with almost constant headaches. The break of two weeks was extremely disruptive to my work flow and the work I started to pile up and the amount of stress that I made myself increased by a lot. The final report needed to be started, the implementation of the web application and the wearable application. I wanted to get an application that I could display and properly describe when it comes to the final report. I started to stress about the results I created and still wanted to create and the creation of the final report. My company supervisor was then able to get me on the right track again to start concentrating on the final report in the first place, as this is for me, the most important deliverable that I need to create. For the future I want to focus on the most important subjects and have a calm mindset when it comes to completing these tasks.

Apart from these issues the second half of the bachelor thesis went better than the first half. The tasks worked on were clearer defined, by myself and accepted by my company supervisor, and more smoothly  completed. Furthermore I would have liked to create a more complete product at this point in time. But that is probably also a lesson I am to take away from this project. There are problems that might occur, and even if those are frustrating to deal with, it is impossible to avoid them and it is easier to accept them and work with what can still be achieved than be frustrated, that the problem existed in the first place. And sometimes some parts might not be completed, but the focus on the important aspects should never be lost in the process.
\end{document}