\documentclass[a4paper]{report}

\usepackage[parfill]{parskip}
\usepackage[pdftex, scale={.8,.8}]{geometry}
\usepackage{titlesec}
\titleformat{\chapter}[hang]
{\normalfont\huge\bfseries}{\thechapter}{1em}{}
\titlespacing*{\chapter}{0pt}{0pt}{20pt}
\begin{document}

\chapter*{Reflection \normalsize{Lukas Rolle - 2310309}}
This reflection is written at the halfway point of my bachelor thesis, and I try to reflect on my personal, professional growth and problem solving process in regards of the way of working.

Up to this point I often found myself not knowing what exactly should the next few tasks be or having the feel, that not enough work has been done in a week. This might be because of the task and how it is intended to be solved. The task itself is relatively free in which kind of wearable I use for the project, what features will become a part of the demo facility and how everything should look like. While that might allow more creative products to exist, it is not the way I am used to work and it took me some time to adjust to this.

The solution I came up with, was to set myself smaller milestones of what I would like to achieve at the end of the week, or at the end of the day. While that might be an obvious solution, and is also conform with a way of working like scrum, it was not as easy from the beginning. I learned from this, that I myself am currently not a person that works at the highest efficiencies when given too much freedom in a task and find it easier to set myself a limit to how much freedom I have while working. I think I was not as productive during the first few weeks of the thesis, as I could have due to that problem, and would like to show myself more of what I am capable of in the second half of the thesis.

Apart from that, I think that the thesis is going to get more interesting from now on. Even though most part until now were interesting from a design perspective, they were too abstract, often without a big deliverable. This lead to rather discouraging results, when at the end of the day nothing could be looked at to say: "This is what I did today". It lead to a state of mind, that was very demoralizing. I think this is a problem in a way of thinking of mine, but I would like to have actual deliverables at the end of the day, that I can be proud of. And I think the coming tasks will be more satisfactory from that point of view.

All in all, even though this reflection might sound a bit negative, I am looking at the future and am hoping to show better results and a better state of mind, than I had until this point.
\end{document}