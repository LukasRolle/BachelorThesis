\chapter{Initial Analysis}\label{cha:initialAnalysis}
The first questions asked in this project were, what logistics processes are existing and which wearables can be used for them. At the beginning of the project not a lot of information was available to use. The only information that was ready in the beginning was the logistics processes that were already researched beforehand. The processes themselves will be discussed in section \ref{sec:processes}. Further on this chapter will discuss the initial list of wearables that could be considered and which requirements will be had for these wearables and why these requirements. Also some requirements towards the reference model and how those came to be will be discussed here.

\section{Wearables}
The initial search for wearables started with wearables that have already been used in projects similar to this. One of the projects that was looked at for this was the pick-by-vision project conducted by Schwerdtfeger. \citep{phdthesis:pickByVision} That is focusing on the application of smart glasses for the order picking process. A case study by DHL shows the usage of the Vuzix M100 and the Google Glasses and is resulting in improved times per task in the picking process. \citep{caseStudy:dhlWearables} 

The results of these previous researches led to the inclusion of a lot of general purpose smartglasses. 

\section{Reference Model}


\begin{itemize}
	\item how were initial wearables been searched
	\item why these requirements towards wearables
	\item process details
	\item how did it get to the reference model?
\end{itemize}