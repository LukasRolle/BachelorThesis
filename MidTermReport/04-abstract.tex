\chapter*{Abstract}\addcontentsline{toc}{chapter}{Abstract}
Small- and medium sized enterprises do have difficulties in trying out new technologies, as they often just do no have the needed resources to spend on the newest technology. The LOGwear project aims to give these companies the possibility to stay competitive in that area, it tries to take generalized logistics processes and combines these with wearables and make the results available to everyone.

This thesis is about the creation of a generalized reference model, that allows everyone a head start on creating their own application with a wearable in the area of logistics. Furthermore a demo facility is planned to be created to allow everyone that is interested  in adopting the wearable technology to get hands-on experience. This demo facility is a physical area, where interested can come to visit and get a demonstration of how a process could look like when using a wearable for it. The environment for the demo facility is trying to mock a real world logistics company as closely as possible, using processes from logistics companies as a base.

This should allow companies, that are interested in adopting new technology, to inform themselves about these technologies easily, find technologies they think are interesting for their way of working, see how the technology actually works in a hands-on environment and then make an educated decision if the adoption of a wearable is something that could help improve their own processes.

\vspace{2cm}

Kleine- und Mittelständische Unternehmen haben oft Probleme damit neue Technologien auszuprobieren, da die dazu nötigen Ressourcen oft nicht vorhanden sind um die neuesten Technologien auszuprobieren. Das LOGwear Projekt versucht diesen Unternehmen die Möglichkeit zu geben in dieser Umgebung trotzdem wettbewerbsfähig zu bleiben. Es versucht standardisierte logistische Prozesse zu nehmen und Sie mit wearables zu verbinden und stellt die Ergebnisse frei zur Verfügung.

Diese Abschlussarbeit beschäftigt sich mit der Erstellung eines standardisierten Referenzmodells, das eine schnellere Erstellung einer eigenen Anwendung mit wearables erlaubt. Weiterhin ist eine Testeinrichtung geplant die kleinen- und mittelständischen Unternehmen die daran interessiert sind, die Möglichkeit gibt  wearables selbst auszuprobieren. Die Testeinrichtung ist eine Umgebung zu der interessierte gehen können, um selbst sehen oder ausprobieren können ob ein bestimmtes wearable etwas für ihre Unternehmensstruktur ist. Es wird versucht mit der Testeinrichtung die tatsächliche Umgebung eines Logistikunternehmens nachzuahmen, in dem man wirkliche logistische Prozesse als Basis für die Testeinrichtung nimmt.

Damit sollte interesierten Unternehmen erlaubt werden, sich auf einfache Art und Weise über neue Technologien zu informieren, Technologien zu finden die in die Unternehmensprozesse passen könnten, zu sehen wie diese Technologien tatsächlich funktionieren und daraufhin eine informierte Entscheidung zu treffen, ob diese Technologie tatsächlich etwas ist, was Ihre Unternehmensprozesse verbessern könnte.