\documentclass{report}
\usepackage[utf8]{inputenc}
\usepackage[a4paper]{geometry}
\usepackage{parskip}
\usepackage{titlesec}
\usepackage{sebirequirements}

%Small Titles
\titleformat{\chapter}[hang] 
{\normalfont\huge\bfseries}{\thechapter}{1em}{} 
\titlespacing*{\chapter}{0pt}{-50pt}{30pt}

%Use Cases without File
\RenewDocumentEnvironment{UseCase}{m m m m m}{% if file should not be contained in the use cases
\begin{reqaspect}{Use Case}{#1}{#2}{#3}{#4}{#5}%
  }{%
  \end{reqaspect}%
}

\author{Lukas Rolle - 2310309}
\title{Use Cases}
\date{\today, Blerick}
\begin{document}
\maketitle
\tableofcontents

\chapter{Use Cases}

\begin{UseCase}{UC-W1}{Get Order (Voice)}{Wearable}{1.0}{LUR}
\Actors{Picking Worker}
\Description{A picking worker equipped with a wearable is using a voice command to get the next order displayed.}
\Preconditions{The wearable has to be equipped with an Input Interface accepting voice and and an Output Interface that is able to return information to the user.}
\Scenario{
	\vspace{0.1\baselineskip}
    \begin{enumerate*}
    	\item The Picking worker gives the voice command "Next Order".
    	\item The Wearable asks the WMS for the next order for the specific worker.
    	\item The WMS sends the Data to the Wearable
    	\item The Wearable displays the Data to the order picker.
	\end{enumerate*}
}
\Extensions{
	\vspace{0.1\baselineskip}
	\begin{enumerate*}
		\item[-]
	\end{enumerate*}
}
\Exceptions{
	\vspace{0.1\baselineskip}
	\begin{enumerate*}
		\item[1.1] The Voice command could not be properly understood. The Wearable does nothing.
		\item[1.2] The Order Picker is already on an Order and that one is unfinished, the command is ignored.
		\item[1.3] Another Voice command is understood, that one is executed.
		\item[3.1] The WMS did not find a next order. The order picker is informed about that.
		\item[4.1] There is an error in the format that was received. The data that could be understood is still displayed and the order picker is informed, that there might be an error with the order.
	\end{enumerate*}
}
\Result{The Order Picker has received the next order.}
\end{UseCase}


\begin{UseCase}{UC-W2}{Order Confirmation}{Wearable}{1.0}{LUR}
\Actors{Picking Worker}
\Description{A picking worker equipped with a wearable is using an input interface to confirm an order.}
\Preconditions{The order picker has finished picking the order and wants to confirm that everything is correct.}
\Scenario{
	\vspace{0.1\baselineskip}
    \begin{enumerate*}
    	\item The order picker uses an input interface on the Wearable to confirm the Order.
    	\item The wearable is processing the request.
    	\item The wearable is sending the Confirmation to the WMS.
    	\item The WMS sends, that the confirmation was successfully received.
    	\item The wearable displays the user, that the confirmation was received by the WMS.
    \end{enumerate*}
}
\Extensions{
	\vspace{0.1\baselineskip}
	\begin{enumerate*}
		\item[1.1] The Wearable is able to check if the confirmation is allowed to be send. See UC-W3.
	\end{enumerate*}
}
\Exceptions{
	\vspace{0.1\baselineskip}
	\begin{enumerate*}
		\item[3.1] The Wearable got an error when processing the request. The order picker is informed about the cause of the error and can try to fix it.
		\item[4.1] The WMS does not answer. The order picker is informed about that and can try contacting the IT.
	\end{enumerate*}
}
\Result{The Order Picker has received the next order.}
\end{UseCase}


\begin{UseCase}{UC-W3}{Order Control}{Wearable}{1.1}{LUR}
\Actors{Picking Worker}
\Description{A picking worker is in the process of picking his order.}
\Preconditions{The wearable has a vision interface or is in another way able to control what the order picker is doing.}
\Scenario{
	\vspace{0.1\baselineskip}
    \begin{enumerate*}
    	\item The order picker scans an item.
    	\item The wearable checks what is being packed.
    	\item The order picker is putting the item on his hand pallet truck.
    	\item The wearable is counting the amount of items put onto the hand pallet truck.
    	\item The order picker scans a new item.
    	\item The wearable detects a new item and ends the counting process for the last item.
    	\item The wearable informs the order picker that everything went correctly with the last item.
    \end{enumerate*}
}
\Extensions{
	\vspace{0.1\baselineskip}
	\begin{enumerate*}
		\item[-]
	\end{enumerate*}
}
\Exceptions{
	\vspace{0.1\baselineskip}
	\begin{enumerate*}
		\item[7.1] The order picker could have counted wrongly and the wearable is informing him about it. After the order picker has checked the quantity on the Truck, go back to 1. with the item started with.
		\item[7.2] The wearable could have counted wrongly and the wearable is informing the order picker as if he counted wrong. The order picker can check the hand pallet truck for the item and confirm the right quantity.
		\item[7.3] The order picker could have picked the wrong quantity and the wearable could have counted wrong, resulting in the wearable saying the order picker has picked the right amount. An error is made.
	\end{enumerate*}
}
\Result{The Order Picker completed a part of the order and the wearable confirmed the quantity of that part.}
\end{UseCase}



\chapter{Communication Layer}
The communication Layer exists due to the possibilities it gives and the need for it depending on the wearable. This means that some enterprises might have the need to do something with the data before entering that into the WMS or something similar, therefore a communication layer allows to add something like a transformation of data at that point and not add more load to the wearable application. On the other hand some wearables might not have the possibilities to directly connect to the API of the WMS. For example some wearable are only able to use radio frequencies to send data and this layer can be used to catch that data, transform that to usable data and send that to the WMS.

\begin{UseCase}{UC-C1}{Receive Message from Wearable}{Communication Layer}{1.0}{LUR}
\Actors{Server}
\Description{The communication layer on the server is receiving a message and sends that further into the system.}
\Preconditions{The wearable has to be able to send data to the communication layer via an internet connection.}
\Scenario{
	\vspace{0.1\baselineskip}
    \begin{enumerate*}
    	\item Receive an incoming message from the wearable.
    	\item Establish a connection to the WMS API.
    	\item Send received data to the WMS.
    \end{enumerate*}
}
\Extensions{
	\vspace{0.1\baselineskip}
	\begin{enumerate*}
		\item[1.1] After receiving a message, the message might have to be converted to be properly understood by the API, that should be done here.
		\item[1.2] When the enterprise wants to do something with the data, it should be done here.
		\item[1.3] When data send has to be validated, it should be done here.
		\item[3.1] The communication layer receives an answer from the WMS, initiate UC-C2.
	\end{enumerate*}
}
\Exceptions{
	\vspace{0.1\baselineskip}
	\begin{enumerate*}
		\item[1.1] Connection could not be established.
		\item[1.2] The wearable send invalid or corrupt data, ask wearable to resend data.
		\item[2.1] Connection could not be established.
	\end{enumerate*}
}
\Result{A message from the wearable has been send to the WMS.}
\end{UseCase}


\begin{UseCase}{UC-C2}{Send Message to Wearable}{Communication Layer}{1.0}{LUR}
\Actors{Server}
\Description{The communication layer on the server is receiving an answer from the WMS for the wearable.}
\Preconditions{The wearable has send a message to the communication layer that is received by the WMS. Furthermore the WMS sends an answer to the received message.}
\Scenario{
	\vspace{0.1\baselineskip}
    \begin{enumerate*}
    	\item Receive an answer from the WMS.
    	\item Establish a connection to the Wearable.
    	\item Send answer to wearable.
    \end{enumerate*}
}
\Extensions{
	\vspace{0.1\baselineskip}
	\begin{enumerate*}
		\item[1.1] After receiving a message, the message might have to be converted to be properly understood by the wearable, that should be done here.
		\item[1.2] When the enterprise wants to do something with the data, it should be done here.
		\item[1.3] When data send has to be validated, it should be done here.
	\end{enumerate*}
}
\Exceptions{
	\vspace{0.1\baselineskip}
	\begin{enumerate*}
		\item[1.1] Connection could not be established.
		\item[2.1] Connection could not be established.
	\end{enumerate*}
}
\Result{A message from the wearable has been send to the WMS.}
\end{UseCase}


\end{document}